\documentclass[10pt,pdf,utf8,russian,aspectratio=169]{beamer}
\usepackage{cmap}
\usepackage[T2A]{fontenc}
%\usepackage[utf8x]{inputenc}
\usepackage[russian,english]{babel}
\usepackage{subfig}
\usepackage{graphicx}
\usepackage{multicol}
\usepackage{cancel}
\usepackage{tabularx}
\usepackage{xargs}      
\usepackage{mathbbol}
\usepackage{amssymb}  
\usepackage{tikz} 
\DeclareSymbolFontAlphabet{\mathbb}{AMSb}%
\DeclareSymbolFontAlphabet{\amsmathbb}{bbold}%



\usetikzlibrary{arrows,automata}
\usetikzlibrary{positioning}


       % AMS Math

%
% Choose how your presentation looks.
%
% For more themes, color themes and font themes, see:
% http://deic.uab.es/~iblanes/beamer_gallery/index_by_theme.html
%
\mode<presentation>
{
  \usetheme{Boadilla}      % or try Darmstadt, Madrid, Warsaw, ...
  \usecolortheme{seagull} % or try albatross, beaver, crane, ..

  \usefonttheme{structurebold}  % or try serif, structurebold, ...
  \setbeamertemplate{navigation symbols}{}
  \setbeamertemplate{caption}[numbered]
} 

\captionsetup[subfloat]{labelformat=empty}
\title[Структура]{Выбор структуры модели глубокого обучения}
\author{Бахтеев Олег}
\institute{МФТИ}
\date{20.11.2019}
%\renewcommand{\headrulewidth}{0pt}
\DeclareMathOperator*{\argmin}{arg\,min}
\DeclareMathOperator*{\argmax}{arg\,max}
\DeclareUnicodeCharacter{00A0}{ } % При наборе текста с планшета появляются невидимые символы. ЭТо костыль.
\begin{document}
% nb: очень не люблю макросы. Но что поделать 
% https://stackoverflow.com/questions/1509799/how-to-replace-latex-macros-with-their-definitions-using-latex
\newcommand{\D}{\mathfrak{D}}
\newcommand{\x}{\mathbf{x}}
\newcommand{\X}{\mathbf{X}}
\newcommand{\y}{\mathbf{y}}
\newcommand{\Xb}{\mathbb{X}}
\newcommand{\yb}{\mathbb{Y}}
\newcommand{\F}{\mathfrak{F}}



\newcommand{\w}{\mathbf{w}}
\newcommand{\Wb}{\mathbb{W}}
\newcommand{\Uw}{U_\mathbf{w}}

\newcommand{\Gam}{\boldsymbol{\Gamma}}
\newcommand{\Gb}{\amsmathbb{\Gamma}}
\newcommand{\UG}{U_{\boldsymbol{\Gamma}}}

\newcommand{\h}{\mathbf{h}}
\newcommand{\Hb}{\mathbb{H}}
\newcommand{\Uh}{U_{\mathbf{h}}}

\newcommand{\teta}{\boldsymbol{\theta}}
\newcommand{\Tetab}{\amsmathbb{\Theta}}
\newcommand{\Uteta}{U_{\boldsymbol{\theta}}}

\newcommand{\tetaw}{\boldsymbol{\theta}_\mathbf{w}}
\newcommand{\Tetawb}{\amsmathbb{\Theta}_\mathbf{w}}
\newcommand{\Utetaw}{U_{\boldsymbol{\theta}_\mathbf{w}}}
\newcommand{\tetaG}{\boldsymbol{\theta}_{\boldsymbol{\Gamma}}}
\newcommand{\TetaGb}{\amsmathbb{\Theta}_{\boldsymbol{\Gamma}}}
\newcommand{\UtetaG}{U_{\boldsymbol{\theta}_{\boldsymbol{\Gamma}}}}

\newcommand{\lam}{\boldsymbol{\lambda}}
\newcommand{\Lamb}{\amsmathbb{\Lambda}}
\newcommand{\Ulam}{U_{\boldsymbol{\lambda}}}

%\newcommand{\prior}{p(\mathbf{w}, \boldsymbol{\Gamma}|\mathbf{h},\boldsymbol{\lambda})}
\newcommandx{\prior}[4][1=\mathbf{w},2=\boldsymbol{\Gamma},3=\mathbf{h},4=\boldsymbol{\lambda},usedefault]{p(#1,#2|#3,#4)}
\newcommandx{\priorh}[2][1=\mathbf{h}, 2=\boldsymbol{\lambda},usedefault]{p(#1|#2)}
\newcommandx{\priorG}[3][1=\boldsymbol{\Gamma}, 2= \mathbf{h}, 3=\boldsymbol{\lambda},usedefault]{p(#1|#2,#3)}
\newcommandx{\priorw}[4][1=\mathbf{w},2=\boldsymbol{\Gamma},3=\mathbf{h},4=\boldsymbol{\lambda},usedefault]{p(#1|#2,#3,#4)}


\newcommand{\post}{p(\mathbf{w}, \boldsymbol{\Gamma}|\mathbf{y}, \mathbf{X}, \mathbf{h},\boldsymbol{\lambda})}
\newcommand{\posth}{p(\mathbf{h}|\mathbf{y}, \mathbf{X},\boldsymbol{\lambda})}
\newcommand{\postG}{p(\boldsymbol{\Gamma}|\mathbf{y}, \mathbf{X}, \mathbf{h},\boldsymbol{\lambda})}
\newcommand{\postw}{p(\mathbf{w}|\mathbf{y}, \mathbf{X}, \boldsymbol{\Gamma}, \mathbf{h},\boldsymbol{\lambda})}


\newcommandx{\q}[1][1=\boldsymbol{\theta}, usedefault]{q(\mathbf{w}, \boldsymbol{\Gamma}|#1)}
\newcommandx{\qG}[2][1=\boldsymbol{\Gamma},2=\boldsymbol{\theta}_{\boldsymbol{\Gamma}},usedefault]{q_{\boldsymbol{\Gamma}}(#1|#2)}
\newcommandx{\qw}[3][1=\mathbf{w}, 2=\boldsymbol{\Gamma},3=\boldsymbol{\theta}_\mathbf{w},usedefault]{q_\mathbf{w}(#1|#2,#3)}


\newcommandx{\LL}[4][1=\mathbf{y},2=\mathbf{X},3=\mathbf{w},4=\boldsymbol{\Gamma},usedefault]{p(#1|#2,#3,#4)}

\newcommand{\EV}{p(\mathbf{y}|\mathbf{X}, \mathbf{h},\boldsymbol{\lambda})}

\newcommandx{\Loss}[5][1=\boldsymbol{\theta},2=\mathbf{y},3=\mathbf{X},4=\mathbf{h},5=\boldsymbol{\lambda},usedefault]{L(#1 |#2,#3,#4,#5)}
\newcommandx{\Val}[5][1=\mathbf{h},2=\mathbf{y},3=\mathbf{X},4=\boldsymbol{\theta},5=\boldsymbol{\lambda},usedefault]{Q(#1|#2,#3,#4,#5)}

% прочее
\newcommand{\model}{\mathbf{f}}
\newcommand{\A}{\mathbf{A}}
\newcommand{\s}{\mathbf{s}}
\newcommand{\g}{\boldsymbol{\gamma}}
\newcommand{\E}{\mathsf{E}}
\newcommand{\KL}[2]{D_\text{KL}\bigl(#1 || #2\bigr)}

\newcommand{\lamT}{\lambda_{\text{temp}}}
\newcommand{\lamLL}{\lambda_\text{likelihood}^\text{Q}}
\newcommand{\lamCL}{\lambda_\text{prior}^\text{L}}
\newcommand{\lamCQ}{\lambda_\text{prior}^\text{Q}}
\newcommand{\lamS}{\boldsymbol{\lambda}_\text{struct}^\text{Q}}
\newcommandx{\TLoss}[6][1=\boldsymbol{\theta},2=L,3=\mathbf{y}, 4=\mathbf{X}, 5=\mathbf{h},6=\boldsymbol{\lambda},usedefault]{T(#1|#2,#3,#4,#5,#6)}
\newcommandx{\TVal}[6][1=\mathbf{h},2=Q,3=\mathbf{y}, 4=\mathbf{X}, 5=\boldsymbol{\teta},6=\boldsymbol{\lambda},usedefault]{T(#1|#2,#3,#4,#5,#6)}
%\newcommand{\log}{\text{log}~}




\begin{frame}
  \titlepage
\end{frame}

\begin{frame}{Резюме прошлых семинаров}
\textbf{Заданы: }
\begin{itemize}
\item Вариационное распределение $\qw$ с параметрами $\teta$;
\item Априорное распределение $\priorw$ с параметрами $\h$;
\item Функция потерь $L$ и функция валидации $Q$.
\end{itemize}
\vspace{0.2cm}
\textbf{Требуется:} предложить метод выбора структуры модели $\Gam$.\\
\vspace{0.2cm}
\textbf{Вопросы:}
\begin{itemize}
\item Как задать структуру модели?
\item Как провести ее выбор?
\item Какова вероятностная интерпретация структуры?
\end{itemize}

\end{frame}

\begin{frame}{Automatic relevance determination}

\end{frame}

\begin{frame}{Пример: вариационный автокодировщик + ARD}
\textbf{VAE:}
\[
    L = \int_{\mathbf{z}} p(\x|\mathbf{z})p(\mathbf{z})d\mathbf{z}.
\]
\textbf{VAE + ARD:}
\[
    L = \iint_{\mathbf{z}, \boldsymbol{\gamma}} p(\x|\mathbf{z} \cdot \boldsymbol{\gamma})p(\mathbf{z})p(\boldsymbol{\gamma})d\mathbf{z}d\boldsymbol{\gamma}.
\]
\includegraphics[width=\textwidth]{ard_vae.png}

\end{frame}



\begin{frame}{SpikeAndSlab}

\end{frame}

\begin{frame}{Гауссовый процесс для выбора структуры модели}
\end{frame}

\begin{frame}{Индийский буфет, еще один пример}
\end{frame}

\begin{frame}{AdaNet}
\end{frame}


\begin{frame}{Neural Architecture Search}
\includegraphics[width=\textwidth]{nas_scheme.png}
\includegraphics[width=\textwidth]{nas_seq.png}
\end{frame}


\begin{frame}{Neural Architecture Search}
\includegraphics[width=\textwidth]{nas_transfer.png}
\end{frame}


\begin{frame}{Neural Architecture Search: результаты}

\begin{figure}
  \centering
 {\includegraphics[width=\textwidth]{zoph.png}}
\label{fig:1}\qquad
\caption*{Zoph et al., 2017.  Сложность моделей отличается почти в два раза при одинаковом качестве.}
\end{figure}
\end{frame}

\begin{frame}{Neural Architecture Search: постановка задачи}
TODO
\end{frame}

\begin{frame}{DARTS}

\end{frame}


\begin{frame}{DARTS}
\end{frame}



\begin{frame}{Графовое представление модели глубокого обучения}
\footnotesize
Заданы:
\begin{enumerate}
 \item ациклический граф $(V,E)$;
\item для каждого ребра $(j,k) \in E$: вектор базовых дифференцируемых функций  $\mathbf{g}^{j,k} = [\mathbf{g}^{j,k}_0, \dots, \mathbf{g}^{j,k}_{K^{j,k}}]$  мощности $K^{j,k}$;
\item для каждой вершины $v \in V$: дифференцируемая функция агрегации $\textbf{agg}_v$.
\item Функция $\mathbf{f} = \mathbf{f}_{|V|-1}$, задаваемая по правилу 
\begin{equation}
\label{eq:modelfam}
    \mathbf{f}_{v}(\mathbf{w}, \mathbf{x}) = \textbf{agg}_{v}\left(\{ \langle \boldsymbol{\gamma}^{j,k}, \mathbf{g}^{j,k} \rangle \circ  \mathbf{f}_j(\mathbf{x})| j \in \text{Adj}(v_k)\}\right), v \in \{1,\dots,|V|-1\}, \quad \mathbf{f}_0(\mathbf{x}) = \mathbf{x}
\end{equation}
и являющаяся функцией из признакового пространства $\mathbb{X}$ в пространство меток $\mathbb{Y}$ при значениях векторов, $\boldsymbol{\gamma}^{j,k} \in [0,1]^{K^{j,k}}$.
\end{enumerate}

\begin{block}{Определение}
Граф $(V, E)$ со множестом векторов базовых функций $\{\mathbf{g}^{j,k}, (j,k) \in E\}$ и функций агрегаций $\{ \textbf{agg}_v, {v \in V}\}$ назовем \textit{параметрическим семейством моделей} $\mathfrak{F}$.
\end{block}
\begin{block}{Утверждение}
Для любого значения $\boldsymbol{\gamma}^{j,k} \in [0,1]^{K^{j,k}}$ функция $\mathbf{f} \in \mathfrak{F}$ является моделью.
\end{block}
\end{frame}


\begin{frame}{Выбор структуры: двуслойная нейросеть}
\small
Модель $\mathbf{f}$ задана \textbf{структурой}  $\boldsymbol{\Gamma} = [\boldsymbol{\gamma}^{0,1}, {\boldsymbol{\gamma}^{1,2}}].$

\[
    \text{Модель: }\mathbf{f}(\mathbf{x}) = \textbf{softmax}\left((\mathbf{w}^{1,2}_0)^\mathsf{T}{\mathbf{f}_1}(\mathbf{x})\right), \quad \mathbf{f}(\mathbf{x}): \mathbb{R}^n \to [0,1]^{|\mathbb{Y}|}, \quad \mathbf{x} \in \mathbb{R}^n.
\]
\[
\mathbf{f}_1(\mathbf{x}) = {\gamma}^{0,1}_{0}\mathbf{g}^{0,1}_{0}(\mathbf{x}) + {\gamma}^{0,1}_{1}\mathbf{g}^{0,1}_{1}(\mathbf{x}),
\]
где $\mathbf{w} = [\mathbf{w}^{0,1}_0, \mathbf{w}^{0,1}_1, \mathbf{w}^{1,2}_0]^{\text{T}}$ --- матрицы параметров, $\{\mathbf{g}^{0}_{0,1},\mathbf{g}^{1}_{0,1},{\mathbf{g}^{0}_{1,2}\}}$ --- обобщенно-линейные функции скрытых слоев нейросети.

\begin{tikzpicture}[node distance=0.5cm, auto]
  %\tikzstyle{every state}=[fill=red,draw=none,text=white]

  \node (f0)  at (1,6)                  {$\mathbf{f}_0(\mathbf{x}) = \mathbf{x}$};
  %\node (g11) at (6,3)                    {$\mathbf{g}^{1,1}(\mathbf{x})$};% = \text{Conv}(\mathbf{x}, 3, 32, 1)$};
  %\node (g12)  at (6,9)                   {$\mathbf{g}^{1,2}(\mathbf{x})$};% = \text{Conv}(\mathbf{x}, 4, 32, 1)$};
  \node (f1)  at (7,6)                 {$\mathbf{f}_1(\mathbf{x})$};% = \gamma^{1,1}\mathbf{g}^{1,1}(\mathbf{x}) +  \gamma^{1,2}\mathbf{g}^{1,2}(\mathbf{x})$};
  %\node (g21) at (12,6)                   {$\mathbf{g}^{2,1}(\mathbf{x})$};% = \boldsymbol{\sigma}(\mathbf{w}^{2,1}\mathbf{x})$};
  \node (f2)  at (12,6)                   {$\mathbf{f}_2(\mathbf{x})$};% = \gamma^{2,1}\mathbf{g}^{2,1}(\mathbf{x})$};
  \path[->]  (f0) edge [bend left=50] node {$\gamma^{0,1}_0\mathbf{g}^{0,1}_0(\mathbf{x}) = \gamma^{0,1}_0\boldsymbol{\sigma}\left((\mathbf{w}^{0,1}_0)^{\mathsf{T}}\mathbf{x}\right)$}(f1);
  \path[->] (f0)  edge[bend right=50] node[below] {$\gamma^{0,1}_1\mathbf{g}^{0,1}_1(\mathbf{x}) = \gamma^{0,1}_1\boldsymbol{\sigma}\left((\mathbf{w}^{0,1}_1)^{\mathsf{T}}\mathbf{x}\right)$}(f1);
  \path[->] (f1)  edge node {$\gamma^{1,2}_0\mathbf{g}^{1,2}_0(\mathbf{x}) = \gamma^{1,2}_0\textbf{softmax}\left((\mathbf{w}^{1,2}_0)^{\mathsf{T}}\mathbf{x}\right)$}(f2);       
  \draw[->] (f1) to (f2);
 
\end{tikzpicture}

\end{frame}





\begin{frame}{Ограничения на структурные параметры}
Примеры ограничений для одного структурного параметра $\boldsymbol{\gamma}, |\boldsymbol{\gamma}| = 3$.
\begin{figure}
 \begin{minipage}[t]{.45\textwidth}
        \centering
%1 limit
\begin{tikzpicture}[%
x={(1.5cm,0cm)},
y={(0cm,1.5cm)},
z={({0.5*cos(45)},{0.5*sin(45)})},
]

\coordinate (A) at (0,0,0); 
\coordinate (B) at (1,0,0) ;
\coordinate (C) at (1,1,0); 
\coordinate (D) at (0,1,0); 
\coordinate (E) at (0,0,1); 
\coordinate (F) at (1,0,1); 
\coordinate (G) at (1,1,1); 
\coordinate (H) at (0,1,1   );

%Ecken
\node[circle,scale=0.5,fill=black,draw=black](Ap) at (0,0,0){};
\node[circle,scale=0.5,fill=black,draw=black](Bp) at (1,0,0){};
\node[circle,scale=0.5,fill=black,draw=black](Cp) at (1,1,0){};
\node[circle,scale=0.5,fill=black,draw=black](Dp) at (0,1,0){};
\node[circle,scale=0.5,fill=black,draw=black](Ep) at (0,0,1){};
\node[circle,scale=0.5,fill=black,draw=black](Fp) at (1,0,1){};
\node[circle,scale=0.5,fill=black,draw=black](Gp) at (1,1,1){};
\node[circle,scale=0.5,fill=black,draw=black](Hp) at (0,1,1){};
\node[left= 1pt of A]{[0,0,0]};
\node[right= 1pt of B]{[1,0,0]};
\node[right= 1pt of C]{[1,1,0]};
\node[left= 1pt of D]{[0,1,0]};
\node[left= 1pt of E]{[0,0,1]};
\node[right= 1pt of F]{[1,0,1]};
\node[right= 1pt of G]{[1,1,1]};
\node[left= 1pt of H]{[0,1,1]};

%Kanten
\draw[] (A)
-- (B)  node[midway, below]{}
-- (C)      node[midway, right]{}
-- (D)  node[midway, above]{}
-- (A)  node[midway, left]{};
\draw[] (B) -- (F) -- (G) -- (C);
\draw[] (G) -- (H) -- (D);
\draw[densely dashed] (A) -- (E) -- (F);
\draw[densely dashed] (E) -- (H);

\end{tikzpicture}
\caption*{На вершинах куба}
\end{minipage}
\hfill
 \begin{minipage}[t]{.45\textwidth}
        \centering

%2 limit
\begin{tikzpicture}[%
x={(1.5cm,0cm)},
y={(0cm,1.5cm)},
z={({0.5*cos(45)},{0.5*sin(45)})},
]

\coordinate (A) at (0,0,0); 
\coordinate (B) at (1,0,0) ;
\coordinate (C) at (1,1,0); 
\coordinate (D) at (0,1,0); 
\coordinate (E) at (0,0,1); 
\coordinate (F) at (1,0,1); 
\coordinate (G) at (1,1,1); 
\coordinate (H) at (0,1,1   );

%Ecken
\node[left= 1pt of A]{[0,0,0]};
\node[right= 1pt of B]{[1,0,0]};
\node[right= 1pt of C]{};
\node[left= 1pt of D]{[0,1,0]};
\node[left= 1pt of E]{};
\node[right= 1pt of F]{[1,0,1]};
\node[right= 1pt of G]{[1,1,1]};
\node[left= 1pt of H]{[0,1,1]};

%Kanten
\draw[fill=gray] (A)
-- (B)  node[midway, below]{}
-- (C)      node[midway, right]{}
-- (D)  node[midway, above]{}
-- (A)  node[midway, left]{};
\draw[fill=gray] (B) -- (F) -- (G) -- (C);
\draw[fill=gray] (G) -- (H) -- (D);
\draw[fill=gray] (A) -- (E) -- (F);
\draw[fill=gray] (E) -- (H);
\draw[fill=gray] (D) -- (H) -- (G) -- (C);
\end{tikzpicture}
\caption*{Внутри куба}
\end{minipage}
\hfill
 \begin{minipage}[t]{.45\textwidth}
        \centering
%3 limit
\begin{tikzpicture}[%
x={(1.5cm,0cm)},
y={(0cm,1.5cm)},
z={({0.5*cos(45)},{0.5*sin(45)})},
]

\coordinate (A) at (0,0,0); 
\coordinate (B) at (1,0,0) ;
\coordinate (C) at (1,1,0); 
\coordinate (D) at (0,1,0); 
\coordinate (E) at (0,0,1); 
\coordinate (F) at (1,0,1); 
\coordinate (G) at (1,1,1); 
\coordinate (H) at (0,1,1   );

%Ecken
\node[circle,scale=0.5,fill=black,draw=black](Bp) at (1,0,0){};
\node[circle,scale=0.5,fill=black,draw=black](Dp) at (0,1,0){};
\node[circle,scale=0.5,fill=black,draw=black](Ep) at (0,0,1){};
\node[left= 1pt of A]{};
\node[right= 1pt of B]{[1,0,0]};
\node[right= 1pt of C]{};
\node[left= 1pt of D]{[0,1,0]};
\node[left= 1pt of E]{[0,0,1]};
\node[right= 1pt of F]{};
\node[right= 1pt of G]{};
\node[left= 1pt of H]{};

%Kanten
\draw[] (A)
-- (B)  node[midway, below]{}
-- (C)      node[midway, right]{}
-- (D)  node[midway, above]{}
-- (A)  node[midway, left]{};
\draw[] (B) -- (F) -- (G) -- (C);
\draw[] (G) -- (H) -- (D);
\draw[densely dashed] (A) -- (E) -- (F);
\draw[densely dashed] (E) -- (H);

\end{tikzpicture}
\caption*{На вершинах симплекса}
\end{minipage}
\hfill
 \begin{minipage}[t]{.45\textwidth}
        \centering
%4 limit
\begin{tikzpicture}[%
x={(1.5cm,0cm)},
y={(0cm,1.5cm)},
z={({0.5*cos(45)},{0.5*sin(45)})},
]

\coordinate (A) at (0,0,0); 
\coordinate (B) at (1,0,0) ;
\coordinate (C) at (1,1,0); 
\coordinate (D) at (0,1,0); 
\coordinate (E) at (0,0,1); 
\coordinate (F) at (1,0,1); 
\coordinate (G) at (1,1,1); 
\coordinate (H) at (0,1,1   );

%Ecken
\node[left= 1pt of A]{};
\node[right= 1pt of B]{[1,0,0]};
\node[right= 1pt of C]{};
\node[left= 1pt of D]{[0,1,0]};
\node[left= 1pt of E]{[0,0,1]};
\node[right= 1pt of F]{};
\node[right= 1pt of G]{};
\node[left= 1pt of H]{};

%Kanten
\draw[] (A)
-- (B)  node[midway, below]{}
-- (C)      node[midway, right]{}
-- (D)  node[midway, above]{}
-- (A)  node[midway, left]{};
\draw[] (B) -- (F) -- (G) -- (C);
\draw[] (G) -- (H) -- (D);
\draw[densely dashed] (A) -- (E) -- (F);
\draw[densely dashed] (E) -- (H);
\draw[fill=gray] (B) -- (D) -- (E);


\end{tikzpicture}
\caption*{Внутри симплекса}
\end{minipage}

\end{figure}

\end{frame}



\begin{frame}{Репараметризация}
\small
\begin{block}{Определение} Случайную величину  $\psi$ с распределением $q$ с параметрами $\teta_\psi$ назовем репараметризованной через случайную величину $\varepsilon$, чье распределение не зависит от параметров $\teta_\psi$, если:
\[
   \psi = g(\varepsilon, \teta_\psi)
\]
где  $g$ --- некоторая непрерывная функция.
\end{block}

\textbf{Пример}
\[
    \E_{\qw} \log~\LL=  \int_{\w} \log~\LL \qw d\w.
\]
Продифференцируем по параметрам $\tetaw$:
\[
 \nabla_{\tetaw} \E_{\qw} \log \LL = 
\int_{\w}  \log \LL \nabla_{\tetaw}\qw d\w.
\]

Пусть возможна репараметризация:
$
    \w = \mathbf{g}(\boldsymbol{\varepsilon}, \tetaw).
$ 
Тогда:
\[
 \nabla_{\tetaw} \E_{\q} \log \LL = \nabla_{\tetaw} \E_{\boldsymbol{\varepsilon}} \log \LL [][][\mathbf{g}(\boldsymbol{\varepsilon})] =
\]
\[= \int_{\boldsymbol{\varepsilon}}\nabla_{\tetaw} \log\LL [][][\mathbf{g}(\boldsymbol{\varepsilon})] p(\boldsymbol{\varepsilon}) d\boldsymbol{\varepsilon}=\E_{\boldsymbol{\varepsilon}} \nabla_{\tetaw} \log\LL [][][\mathbf{g}(\boldsymbol{\varepsilon})].\]

\end{frame}


\begin{frame}{Reparametrization}
\end{frame}


\begin{frame}{Logit-Normal}
\end{frame}


\begin{frame}{Gumbel-Softmax}
\end{frame}

\begin{frame}{Proposed Method}

\end{frame}


\begin{frame}{Proposed Method}

\end{frame}


\begin{frame}{Properties}
\end{frame}

\begin{frame}{Examples}
\end{frame}

\begin{frame}
\frametitle{Используемые материалы}
\footnotesize
%\begin{enumerate}

%\end{enumerate}
\end{frame}
\end{document}
